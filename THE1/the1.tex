\documentclass[12pt]{article}
\usepackage[utf8]{inputenc}
\usepackage{float}
\usepackage{amsmath}

\usepackage[hmargin=3cm,vmargin=6.0cm]{geometry}
%\topmargin=0cm
\topmargin=-2cm
\addtolength{\textheight}{6.5cm}
\addtolength{\textwidth}{2.0cm}
%\setlength{\leftmargin}{-5cm}
\setlength{\oddsidemargin}{0.0cm}
\setlength{\evensidemargin}{0.0cm}

%misc libraries goes here
\usepackage{fitch}
\usepackage[shortlabels]{enumitem}
\usepackage{amsmath}
\usepackage{mathabx}
\usepackage{logicproof}

\newcommand\+{\mkern2mu}
\let\eps\varepsilon

\begin{document}

\section*{Student Information }
%Write your full name and id number between the colon and newline
%Put one empty space character after colon and before newline
Full Name : Murat Bolu \\
Id Number : 2521300 \\

% Write your answers below the section tags
\section*{Answer 1}
Part \textbf{a)}
\begin{displaymath}
\begin{array}{|c c|c|c|c|c|c|}
p & q & \lnot p & \lnot q & p \land q & \lnot p \lor \lnot q & (p \land q) \leftrightarrow (\lnot p \lor \lnot q) \\
\hline
T & T & F & F & T & F & F \\
T & F & F & T & F & T & F \\
F & T & T & F & F & T & F \\
F & F & T & T & F & T & F \\
\end{array}
\end{displaymath}
\begin{center}
Therefore, it is a contradiction.
\end{center}

\noindent
Part \textbf{b)}
\begin{align*}
p \to ((q \lor \lnot q) \to (p \land q)) &\equiv p \to (T \to (p \land q)) & \text{Negation Law, Table 6} \\
&\equiv p \to (\lnot T \lor (p \land q)) & \text{Table 7, 1\textsuperscript{st} Line} \\
&\equiv p \to (F \lor (p \land q)) & \text{Negation of Truth} \\
&\equiv p \to ((p \land q) \lor F) & \text{Commutative Law, Table 6} \\
&\equiv p \to (p \land q) & \text{Identity Law, Table 6} \\
&\equiv \lnot p \lor (p \land q) & \text{Table 7, 1\textsuperscript{st} Line} \\
&\equiv (\lnot p \lor p) \land (\lnot p \lor q) & \text{Distributive Law, Table 6} \\
&\equiv (p \lor \lnot p) \land (\lnot p \lor q) & \text{Commutative Law, Table 6} \\
&\equiv T \land (\lnot p \lor q) & \text{Negation Law, Table 6} \\
&\equiv (\lnot p \lor q) \land T & \text{Commutative Law, Table 6} \\
&\equiv \lnot p \lor q & \text{Identity Law, Table 6}
\end{align*}

\section*{Answer 2}
\begin{enumerate}[label=\textbf{\alph*)}] % a), b), c), ...
\item $\forall x \; \exists \+ y \; W(x, y)$
\item $\exists \+ y \; \forall x \; (\lnot F(x, y))$
\item $\forall x \; (W(x, P) \to A(a, x))$, where a is Ali.
\item $\exists \+ x \; (W(b, x) \land F(t, x))$, where b is Büşra and t is TÜBİTAK.
\item $\exists \+ x \; \exists \+ y \; \exists \+ z \; (S(x, y) \land S(x, z) \land \neg(y = z))$
\item $\forall x \; \forall y \; \forall z \; ((W(x, z) \land W(y, z)) \to (x = y))$, where $x$ and $y$ are students and $z$ is a project. % \equiv \forall x \; \forall y \; \forall z \; (\lnot W(x, z) \lor \lnot W(y, z) \lor x = y)$ //another solution
\item $\exists \+ x \; \exists y \; \exists z \; (W(y, x) \land W(z, x) \land \lnot(y=z) \land \forall t \; (W(t, x) \to ((y = t) \lor (z = t))))$, where $x$ is a project and $y$, $z$ and $t$ are students. % $\equiv \exists \+ x \; \forall y \; \forall z \; (W(y, x) \land W(z, x) \land \lnot(y=z) \land \forall t \; (\lnot W(t, x) \lor y = t \lor z = t))$ //another solution
\end{enumerate}

\section*{Answer 3}
\begin{logicproof}{2}
    p \to q & Premise \\
    (q \land \lnot r) \to s & Premise \\
    \lnot s & Premise \\
    \begin{subproof}
        p & Assumption \\
        \begin{subproof}
            \lnot r & Assumption \\
            q & $\to$ e, 4, 1 \\
            q \land \lnot r & $\land$ i, 6, 5 \\
            s & $\to$ e, 7, 2 \\
            \bot & $\lnot$ e, 3, 8
        \end{subproof}
        \lnot \lnot r & $\lnot$ i, 6 - 9 \\
        r & $\lnot$ $\lnot$ e, 10
    \end{subproof}
    p \to r & $\to$ i, 4 - 11
\end{logicproof}

\section*{Answer 4}
If we translate the sentences to logical formulas:
\begin{itemize}
\item Ayşe: $p$
\item Barış: $s \to \lnot q$
\item Can: $p \to (q \land r)$
\item Duygu: $r \to s$
\end{itemize}
Now, we need to show that $p, \; p \to (q \land r), \; r \to s \vdash \lnot(s \to \lnot q)$.
\begin{logicproof}{2}
    p & Premise \\
    p \to (q \land r) & Premise \\
    r \to s & Premise \\
    q \land r & $\to$ e, 1, 2 \\
    q & $\land$ e, 4 \\
    r & $\land$ e, 4 \\
    s & $\to$ e, 6, 3 \\
    \begin{subproof}
        s \to \lnot q & Assumption \\
        \lnot q & $\to$ e, 7, 8 \\
        \bot & $\lnot$ e, 5, 9
    \end{subproof}
    \lnot(s \to \lnot q) & $\lnot$ i, 8 - 10
\end{logicproof}

\section*{Answer 5}
\begin{logicproof}{2}
    \forall x \; (P(x) \to (Q(x) \to R(x))) & Premise \\
    \exists \+ x \; (P(x)) & Premise \\
    \forall x \; (\lnot R(x)) & Premise \\
    \begin{subproof}
        P(c) & Assumption \\
        P(c) \to (Q(c) \to R(c)) & $\forall\+$e, 1 \\
        Q(c) \to R(c) & $\to$ e, 4, 5 \\
        \begin{subproof}
            Q(c) & Assumption \\
            R(c) & $\to$ e, 7, 6 \\
            \lnot R(c) & $\forall\+$e, 3 \\
            \bot & $\lnot$ e, 8, 9
        \end{subproof}
        \lnot Q(c) & $\lnot$ i, 7 - 10 \\
        \exists \+ x \; (\lnot Q(x)) & $\exists\+$i, 11
    \end{subproof}
    \exists \+ x \; (\lnot Q(x)) & $\exists\+$e, 2, 4 - 12
\end{logicproof}

\end{document}