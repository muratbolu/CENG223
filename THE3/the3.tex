\documentclass[12pt]{article}
\usepackage[utf8]{inputenc}
\usepackage{float}
\usepackage{amsmath}

\usepackage[hmargin=3cm,vmargin=6.0cm]{geometry}
%\topmargin=0cm
\topmargin=-2cm
\addtolength{\textheight}{6.5cm}
\addtolength{\textwidth}{2.0cm}
%\setlength{\leftmargin}{-5cm}
\setlength{\oddsidemargin}{0.0cm}
\setlength{\evensidemargin}{0.0cm}

%misc libraries goes here
\usepackage{amssymb}
\usepackage[group-separator={,}]{siunitx}
\newcommand{\divides}{\mid}
\newcommand{\notdivides}{\nmid}
\newcommand{\Mod}[1]{\ (\mathrm{mod}\ #1)}
\newcommand{\+}{\mkern2mu}
\newcommand{\uq}[1]{\forall #1}
\newcommand{\eq}[1]{\exists \+ #1}
\newcommand{\bb}[1]{\mathbb{#1}}
\newcommand*{\myPerm}[2]{{}^{#1}\!P_{#2}}%

\begin{document}

\section*{Student Information }
%Write your full name and id number between the colon and newline
%Put one empty space character after colon and before newline
Full Name : Murat Bolu \\
Id Number : 2521300 \\

% Write your answers below the section tags
\section*{Answer 1}

\begin{description}
    \item[Basis:] For $n = 1$, $6^{2n} - 1 = 6^2 - 1 = 35$, and $5 \divides 35$, and $7 \divides 35$.
    \item[Inductive Step:] Assuming that $6^{2n} - 1$ is divisible by both $5$ and $7$, one can show that $6^{2(n+1)} - 1 = 6^{2n+2} - 1$ is divisible by $5$ and $7$ for $n \in \{1, 2, 3, \dotsc\}$. Since $5 \divides 6^{2n} - 1$ and $7 \divides 6^{2n} - 1$, that means $35 \divides 6^{2n} - 1$ since $5$ and $7$ are relatively prime, which is also shown below.
    \begin{align*}
        6^{2n} - 1 &= 5 \cdot x && \text{for some $x \in \mathbb{N}$} \\
        6^{2n} - 1 &= 7 \cdot y && \text{for some $y \in \mathbb{N}$} \\
        5 \cdot x &= 7 \cdot y
    \end{align*}
    Since $x$ and $y$ are natural numbers, $x$ must be divisible by $7$ and $y$ must be divisible by $5$, and we can rewrite them as $x = 7 \cdot k$ and $y = 5 \cdot l$. The other possibility is that they are both zero, which is obviously not possible since $6^{2n} - 1$ is at least $35$.
    \begin{align*}
        6^{2n} - 1 &= 5 \cdot 7 \cdot k && \text{for some $k \in \mathbb{N}$} \\
        6^{2n} - 1 &= 7 \cdot 5 \cdot l && \text{for some $l \in \mathbb{N}$} \\
        6^{2n} - 1 &= 35 \cdot k \\
        6^{2n} - 1 &= 35 \cdot l
    \end{align*}
    There exists $k$ or $l$ such that they are natural numbers, and that means $6^{2n} - 1$ is divisible by $35$.
    \begin{align*}
        6^{2n} - 1 &\equiv 0 && \Mod{35} \\
        6^{2n} &\equiv 1 && \Mod{35} \\
        6^{2n+2} &\equiv 36 && \Mod{35} \\
        6^{2n+2} &\equiv 1 && \Mod{35} \\
        6^{2n+2} - 1 & \equiv 0 && \Mod{35}
    \end{align*}
    Therefore, there exists a natural number $a \in \mathbb{N}$ such that $6^{2n+2} - 1 = 35 \cdot a$. It is obvious that $6^{2n+2} - 1$ is divisible by both $5$ and $7$ since $5 \divides 35a$ and $7 \divides 35a$.

    \hfill $\blacksquare$
\end{description}

\newpage

\section*{Answer 2}

\begin{description}
    \item[Basis:]
    \begin{itemize}
        \item[]
        \item For $n = 0$, $H_0 = 1 \leq 1 = 9^0$.
        \item For $n = 1$, $H_1 = 5 \leq 9 = 9^1$.
        \item For $n = 2$, $H_2 = 7 \leq 81 = 9^2$.
        \item For $n = 3$, $H_3 = 8H_2 + 8H_1 + 9H_0 = 8 \cdot 7 + 8 \cdot 5 + 9 \cdot 1 = 105 \leq 729 = 9^3$.
    \end{itemize}
    \item[Inductive Step:] Assuming that $H_n \leq 9^n$, $\uq{n} \leq k$, one can show that $H_{k+1}$ is less than or equal to $9^{k+1}$. We can replace every $H_n$ with something greater than or equal to itself.
    \begin{align*}
        H_{k+1} &= 8H_k + 8H_{k-1} + 9H_{k-2} \\
        H_{k+1} &\leq 8 \cdot 9^k + 8H_{k-1} + 9H_{k-2} \\
        H_{k+1} &\leq 8 \cdot 9^k + 8 \cdot 9^{k-1} + 9H_{k-2} \\
        H_{k+1} &\leq 8 \cdot 9^k + 8 \cdot 9^{k-1} + 9 \cdot 9^{k-2} \\
        H_{k+1} &\leq 8 \cdot 9^k + 8 \cdot 9^{k-1} + 9^{k-1} \\
        H_{k+1} &\leq 8 \cdot 9^k + 9 \cdot 9^{k-1} \\
        H_{k+1} &\leq 8 \cdot 9^k + 9^{k} \\
        H_{k+1} &\leq 9 \cdot 9^k \\
        H_{k+1} &\leq 9^{k+1}
    \end{align*}

    \hfill $\blacksquare$
\end{description}

\section*{Answer 3}

Let $A$ denote the set that contains 8 digit bit strings that contain four consecutive zeros and $B$ denote the set that contains 8 digit bit strings that contain four consecutive ones. It it clear that the answer we are looking for is $|A \cup B|$ which is equal to the commonly known identity $|A| + |B| - |A \cap B|$. It can be seen that $A \cap B = \{00001111, 11110000\}$ and $|A \cap B| = 2$. Then, we need to find $|A|$ and $|B|$.
\newline
Let $a_n$ denote the number of strings of size $n$ that contain four consecutive zeros and $b_n$ denote the number of strings of size $n$ that contain four consecutive ones. One can see that $|A| = a_8$ and $|B| = b_8$.
\newline
$a_n$ can be written as a recurrence relation where $a_n = 2 \cdot a_{n-1} + 2^{n-5} - a_{n-5}$ and $a_0 = a_1 = a_2 = a_3 = 0$, $a_4 = 1$. The initial conditions are obvious, and the reason for recurrence relation is that we can append zero or one to a valid digit of size $n - 1$ to get a valid digit of size $n$, hence $2 \cdot a_{n-1}$. We can also take a invalid digit of size $n - 5$ and append $10000$ to it, invalid digits are all digits minus the valid digits, hence $2^{n-5} - a_{n-5}$. One can calculate $a_5 = 3$, $a_6 = 8$, $a_7 = 20$, and $a_8 = 48$.
\newline
The same argument applies to $b_n$ where the bits are flipped, so $b_8 = 48$. Therefore $|A| = 48$, $|B| = 48$, and $|A \cup B| = 48 + 48 - 2 = 94$.

\section*{Answer 4}

One can choose the stars, habitable planets, and inhabitable planets such that they are chosen in the order the closest to the star to the farthest. One needs to choose $1$ star from $10$ stars, $2$ habitable planets from $20$ habitable planets, $8$ inhabitable planets from $80$ inhabitable planets in order. Therefore, there are $\myPerm{10}{1} \cdot \myPerm{20}{2} \cdot \myPerm{80}{8} = \num[group-separator={,}]{4441354491974400000}$ possible choices for different stars and planets. For different orderings between planets, if we consider the habitable planets same with each other and inhabitable planets same with each other, there are $6$ possibilities.
\begin{gather*}
    s \quad uuhuuuuuuh \\
    s \quad uhuuuuuuuh \\
    s \quad huuuuuuuuh \\
    s \quad uhuuuuuuhu \\
    s \quad huuuuuuuhu \\
    s \quad huuuuuuhuu
\end{gather*}
One may also consider the block $huuuuuuh$ with six $u$'s in the middle unmoving and two remaining $u$'s moving. In that case, $h$'s act like separators and two remaining $u$'s can be placed in any three positions. There are $4$ objects in total, two separators and two $u$'s. Both pairs have identical objects, so the permutation of these objects must be divided by $2!$ two times. Thus, the possible orderings are $\tfrac{4!}{2! \cdot 2!} = 6$.
\begin{gather*}
    s \quad \_|\_|\_ \\
    s \quad \_|uu|\_ \\
\end{gather*}
Thus, there are $6$ different ways to order the planets. In total, there are $\myPerm{10}{1} \cdot \myPerm{20}{2} \cdot \myPerm{80}{8} \cdot 6 = \num[group-separator={,}]{26648126951846400000}$ different ways to create a galaxy.

\section*{Answer 5}

Let the robot start from position $0$, much like array indexing. So when the robot jumps one cell to the right, it is at position $1$.
\paragraph{a)}
Let $a_n$ be the number of ways the robot can reach the $n^\text{th}$ cell. The recurrence relation for $a_n$ is $a_n = a_{n-1} + a_{n-2} + a_{n-3}$ for $n \geq 4$ with initial conditions $a_1 = 1$, $a_2 = 2$, $a_3 = 4$.

\paragraph{b)}
Initial conditions are $a_1 = 1$, $a_2 = 2$, $a_3 = 4$ since there is only one way to reach $1^\text{st}$ cell, by jumping one cell at once. There are two ways to reach $2^\text{nd}$ cell, by jumping one by one two times and by jumping two cells at once. There are four ways to reach $3^\text{rd}$ cell, by jumping one by one three times, by jumping two cells and one cell, by jumping one cell and two cells, and by jumping three cells at once.

\paragraph{c)}
The robot can move to the $9^\text{th}$ cell to the right in $a_9$ different ways, which can be calculated bottom-up.
\begin{align*}
    a_4 &= a_3 + a_2 + a_1 = 4 + 2 + 1 = 7 \\
    a_5 &= a_4 + a_3 + a_2 = 7 + 4 + 2 = 13 \\
    a_6 &= a_5 + a_4 + a_3 = 13 + 7 + 4 = 24 \\
    a_7 &= a_6 + a_5 + a_4 = 24 + 13 + 7 = 44 \\
    a_8 &= a_7 + a_6 + a_5 = 44 + 24 + 13 = 81 \\
    a_9 &= a_8 + a_7 + a_6 = 81 + 44 + 24 = 149
\end{align*}
The robot can move to the $9^\text{th}$ cell to the right in $149$ different ways.

\end{document}
