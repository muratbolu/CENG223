\documentclass[12pt]{article}
\usepackage[utf8]{inputenc}
\usepackage{float}
\usepackage{amsmath}

\usepackage[hmargin=3cm,vmargin=6.0cm]{geometry}
%\topmargin=0cm
\topmargin=-2cm
\addtolength{\textheight}{6.5cm}
\addtolength{\textwidth}{2.0cm}
%\setlength{\leftmargin}{-5cm}
\setlength{\oddsidemargin}{0.0cm}
\setlength{\evensidemargin}{0.0cm}

\usepackage{braket}
\usepackage{amssymb}
\usepackage{tikz}
\newcommand{\+}{\mkern2mu}
\newcommand{\ndiv}{\hspace{-4pt}\not|\hspace{2pt}}

\begin{document}

\addtolength\jot{5pt}

\section*{Student Information }
%Write your full name and id number between the colon and newline
%Put one empty space character after colon and before newline
Full Name : Murat Bolu \\
Id Number : 2521300 \\

% Write your answers below the section tags
\section*{Answer 1}
Let $\Braket{a_0, a_1, a_2, a_3, \cdots, a_n, \cdots}$ be the generating function $A(x)$. Then, $A(x) = \displaystyle\sum_{n=0}^{\infty} a_n x^n$.
\begin{align*}
\sum_{n=2}^{\infty} a_n x^n &= \sum_{n=2}^{\infty} (3a_{n-1} + 4a_{n-2}) \cdot x^n \\
&= \sum_{n=2}^{\infty} 3a_{n-1} x^n + 4a_{n-2} x^n \\
&= \sum_{n=2}^{\infty} 3a_{n-1} x^n + \sum_{n=2}^{\infty} 4a_{n-2} x^n \\
&= 3 \sum_{n=2}^{\infty} a_{n-1} x^n + 4 \sum_{n=2}^{\infty} a_{n-2} x^n \\
&= 3x \sum_{n=2}^{\infty} a_{n-1} x^{n-1} + 4x^2 \sum_{n=2}^{\infty} a_{n-2} x^{n-2} \\
&= 3x \sum_{n=1}^{\infty} a_n x^n + 4x^2 \sum_{n=0}^{\infty} a_n x^n \\
\sum_{n=2}^{\infty} a_n x^n &= 3x \sum_{n=1}^{\infty} a_n x^n + 4x^2 \sum_{n=0}^{\infty} a_n x^n \\
A(x) - a_1 \cdot x - a_0 &= 3x \cdot (A(x) - a_0) + 4x^2 \cdot A(x) \\
A(x) - a_1 \cdot x - a_0 &= 3x \cdot A(x) - 3x \cdot a_0 + 4x^2 \cdot A(x) \\
A(x) - 3x \cdot A(x) - 4x^2 \cdot A(x) &= a_1 \cdot x + a_0 - 3x \cdot a_0 \\
A(x) \cdot (1 - 3x - 4x^2) &= a_1 \cdot x + a_0 \cdot (1 - 3x) \\
A(x) &= \frac{a_1 \cdot x + a_0 \cdot (1 - 3x)}{1 - 3x - 4x^2} \\
&= \frac{a_1 \cdot x + a_0 \cdot (1 - 3x)}{(1-4x) \cdot (1+x)}
\end{align*}
Plug in values for $a_0$ and $a_1$:
\begin{align*}
A(x) &= \frac{x + 1 \cdot (1 - 3x)}{(1-4x) \cdot (1+x)} \\
&= \frac{1 - 2x}{(1-4x) \cdot (1+x)}
\end{align*}

\noindent
Partial fractions:
\begin{align*}
\frac{1 - 2x}{(1-4x) \cdot (1+x)} &= \frac{\xi}{1-4x} + \frac{\varphi}{1+x} \\
&= \frac{\xi \cdot (1+x) + \varphi \cdot (1-4x)}{(1-4x) \cdot (1+x)} \\
&= \frac{\xi + \xi \cdot x + \varphi - 4 \varphi \cdot x}{(1-4x) \cdot (1+x)} \\
\frac{1 - 2x}{(1-4x) \cdot (1+x)} &= \frac{\xi + \varphi + x \cdot (\xi  - 4 \varphi)}{(1-4x) \cdot (1+x)}
\end{align*}
We can solve the system of linear equations:
\begin{align*}
\xi + \varphi &= 1 \\
\xi -4 \varphi &= -2 \\
5 \varphi &= 3 \\
\varphi &= \tfrac{3}{5} \\
\xi &= \tfrac{2}{5}
\end{align*}
So the equation becomes:
\begin{align*}
A(x) &= \frac{\tfrac{2}{5}}{1-4x} + \frac{\tfrac{3}{5}}{1+x} \\
&= \frac{2}{5} \cdot \frac{1}{1-4x} + \frac{3}{5} \cdot \frac{1}{1+x}
\end{align*}
\begin{align*}
\frac{1}{1-4x} &\leftrightarrow \Braket{4^0, 4^1, 4^2, \cdots, 4^n, \cdots} \\
\frac{2}{5} \cdot \frac{1}{1-4x} &\leftrightarrow \Braket{\tfrac{2}{5} \cdot 4^0, \tfrac{2}{5} \cdot 4^1, \tfrac{2}{5} \cdot 4^2, \cdots, \tfrac{2}{5} \cdot 4^n, \cdots} \\
&\leftrightarrow \Braket{\frac{2}{5}, \frac{8}{5}, \frac{32}{5}, \cdots, \frac{2^{2n+1}}{5}, \cdots}
\end{align*}
\begin{align*}
\frac{1}{1+x} &\leftrightarrow \Braket{1, -1, 1, \cdots, (-1)^n, \cdots} \\
\frac{3}{5} \cdot \frac{1}{1+x} &\leftrightarrow \Braket{\frac{3}{5}, -\frac{3}{5}, \frac{3}{5}, \cdots, \frac{3}{5} \cdot (-1)^n, \cdots}
\end{align*}
\begin{align*}
A(x) &\leftrightarrow \Braket{\frac{2}{5}, \frac{8}{5}, \frac{32}{5}, \cdots, \frac{2^{2n+1}}{5}, \cdots} + \Braket{\frac{3}{5}, -\frac{3}{5}, \frac{3}{5}, \cdots, \frac{3}{5} \cdot (-1)^n, \cdots} \\
A(x) &\leftrightarrow \Braket{1, 1, 7, \cdots, \frac{2^{2n+1} + 3(-1)^n}{5}, \cdots}
\end{align*}
Therefore, $a_n = \dfrac{2^{2n+1} + 3(-1)^n}{5}$.

\section*{Answer 2}
\subsection*{a) }
Let $\Braket{2, 5, 11, 29, 83, 245, \cdots}$ be the generating function $F(x)$.
\begin{align*}
F(x) &\leftrightarrow \Braket{2, 5, 11, 29, 83, 245, \cdots} \\
&\leftrightarrow \Braket{0+2, 3+2, 9+2, 27+2, 81+2, 243+2, \cdots} \\
&\leftrightarrow \Braket{0, 3, 9, 27, 81, 243, \cdots} + \Braket{2, 2, 2, 2, 2, 2, \cdots} \\
&\leftrightarrow \Braket{0, 3, 9, 27, 81, 243, \cdots} + 2 \cdot \Braket{1, 1, 1, 1, 1, 1, \cdots} \\
&\leftrightarrow \Braket{1-1, 3, 9, 27, 81, 243, \cdots} + 2 \cdot \Braket{1, 1, 1, 1, 1, 1, \cdots} \\
&\leftrightarrow \Braket{1, 3, 9, 27, 81, 243, \cdots} - \Braket{1, 0, 0, 0, 0, 0, \cdots} + 2 \cdot \Braket{1, 1, 1, 1, 1, 1, \cdots}
\end{align*}
\begin{align*}
F(x) &= \frac{1}{1-3x} - 1 + 2 \cdot \frac{1}{1-x} \\
&= \frac{1-1+3x}{1-3x} + \frac{2}{1-x} \\
&= \frac{3x}{1-3x} + \frac{2}{1-x} \\
&= \frac{3x \cdot (1-x) + 2 \cdot (1-3x)}{(1-3x) \cdot (1-x)} \\
&= \frac{3x - 3x^2 + 2 - 6x}{1 - 4x + 3x^2} \\
F(x) &= \frac{2 - 3x - 3x^2}{1 - 4x + 3x^2}
\end{align*}

\subsection*{b) }
\begin{align*}
G(x) &= \frac{7-9x}{1-3x+2x^2} \\
&= \frac{7-9x}{(1-2x)(1-x)}
\end{align*}
\begin{align*}
\frac{7-9x}{(1-2x)(1-x)} &= \frac{\xi}{1-2x} + \frac{\varphi}{1-x} \\
&= \frac{\xi \cdot (1-x) + \varphi \cdot (1-2x)}{(1-2x)(1-x)} \\
&= \frac{\xi - \xi \cdot x + \varphi - 2 \varphi \cdot x}{(1-2x)(1-x)} \\
&= \frac{\xi + \varphi - \xi \cdot x - 2 \varphi \cdot x}{(1-2x)(1-x)} \\
\frac{7-9x}{(1-2x)(1-x)} &= \frac{\xi + \varphi + x \cdot (- \xi - 2 \varphi)}{(1-2x)(1-x)}
\end{align*}
\begin{align*}
\xi + \varphi &= 7 \\
-\xi - 2\varphi &= -9 \\
-\varphi &= -2 \\
\varphi &= 2 \\
\xi &= 5
\end{align*}
\begin{align*}
G(x) &= \frac{5}{1-2x} + \frac{2}{1-x} \\
&= 5 \cdot \frac{1}{1-2x} + 2 \cdot \frac{1}{1-x}
\end{align*}
\begin{align*}
\frac{1}{1-2x} &\leftrightarrow \Braket{1, 2, 4, 8, \cdots, 2^n, \cdots} \\
5 \cdot \frac{1}{1-2x} &\leftrightarrow \Braket{5, 10, 20, 40, \cdots, 5 \cdot 2^n, \cdots} \\
\frac{1}{1-x} &\leftrightarrow \Braket{1, 1, 1, 1, \cdots, 1, \cdots} \\
2 \cdot \frac{1}{1-x} &\leftrightarrow \Braket{2, 2, 2, 2, \cdots, 2, \cdots}
\end{align*}
\begin{align*}
5 \cdot \frac{1}{1-2x} + 2 \cdot \frac{1}{1-x} &\leftrightarrow \Braket{5, 10, 20, 40, \cdots, 5 \cdot 2^n, \cdots} + \Braket{2, 2, 2, 2, \cdots, 2, \cdots} \\
G(x) &\leftrightarrow \Braket{7, 12, 22, 42, \cdots, 5 \cdot 2^n + 2, \cdots} \\
\end{align*}
Therefore, $g_n = 5 \cdot 2^n + 2$.

\section*{Answer 3}
\subsection*{a) }
$R$ can be a equivalence relation if it is reflexive, symmetric and transitive.
$R$ is not reflexive, it is symmetric and it is not transitive.
$R$ is not reflexive since $1 \in \mathbb{Z}$, but $1R1$ is not true since there does not exist a right triangle with two edges of size one and the other edge being an integer.
In a right triangle, hypotenuse is always the longest side of the triangle.
If the hypotenuse is $1$, by Pythagorean theorem $x^2 + 1^2 = 1^2 \to x^2 = 0 \to x = 0$, therefore one of the sides is zero and the shape cannot be a triangle.
If the hypotenuse is not $1$, by Pythagorean theorem $1^2 + 1^2 = x^2 \to 2 = x^2$ and by quadratic formula $x=\sqrt{2}$ since sides are always positive.
Therefore, x is not an integer.
The relation $1R1$ is not satisfied either way.
Since $1R1$ is not satisfied, $R$ is not reflexive, and since $R$ is not reflexive, it cannot be an equivalence relation.

\subsection*{b) }
$R$ can be a equivalence relation if it is reflexive, symmetric and transitive.
$R$ is reflexive, symmetric, and transitive.
\begin{itemize}
\item $R$ is reflexive since for all pairs,
    \begin{align*}
    (x_1, y_1)R(x_1, y_1) &\leftrightarrow 2x_1 + y_1 = 2x_1 + y_1 \\
    &\leftrightarrow 0 = 0 \\
    &\leftrightarrow T
    \end{align*}
    Which means $(x_1, y_1)R(x_1, y_1)$ is always true.
\item $R$ is symmetric since for all pairs,
    \begin{align*}
    (x_1, y_1)R(x_2, y_2) &\to 2x_1 + y_1 = 2x_2 + y_2 \\
    &\to 2x_2 + y_2 = 2x_1 + y_1 \\
    &\to (x_2, y_2)R(x_1, y_1)
    \end{align*}
\item $R$ is transitive since for all pairs,
    \begin{align*}
    (x_1, y_1)R(x_2, y_2) \land (x_2, y_2)R(x_3, y_3) &\to (2x_1 + y_1 = 2x_2 + y_2) \land (2x_2 + y_2 = 2x_3 + y_3) \\
    &\to 2x_1 + y_1 = 2x_2 + y_2 = 2x_3 + y_3 \\
    &\to 2x_1 + y_1 = 2x_3 + y_3 \\
    &\to (x_1, y_1)R(x_3, y_3)
    \end{align*}
\end{itemize}
For the equivalence class of $(1, -2)$:
\begin{align*}
(1, -2)R(x, y) &\leftrightarrow 2 \cdot 1 - 2 = 2x + y \\
&\leftrightarrow 0 = 2x + y \\
&\leftrightarrow -2x = y
\end{align*}
Therefore, $(1, -2)R(x, -2x)$ for all $x \in \mathbb{R}$.
The equivalence class $[(1, -2)]$ is equal to the set $\{(x, -2x) \mid x \in \mathbb{R}\}$.
If we take the pairs like points on a Cartesian coordinate system, this equivalence class represents the line passing through $(1, -2)$ with the slope $-2$. We can find the slope like this:
\begin{align*}
(x_1, y_1)R(x_2, y_2) &\leftrightarrow 2x_1 + y_1 = 2x_2 + y_2 \\
&\leftrightarrow y_1 - y_2 = 2x_2 - 2x_1 \\
&\leftrightarrow -(y_2 - y_1) = 2(x_2 - x_1) \\
&\leftrightarrow \frac{y_2 - y_1}{x_2 - x_1} = -2 \\
\end{align*}
In fact, this relation $R$ partitions $\mathbb{R}^2$ into lines with slope $-2$.

\section*{Answer 4}

\subsection*{a) }
\begin{center}
\begin{tikzpicture}
    \node (2) at (1,-2) {$2$};
    \node (5) at (-1, -2) {$5$};
    \node (10) at (0, 0) {$10$};
    \node (18) at (2, -0) {$18$};
    \node (60) at (0, 2) {$60$};
    \draw (18) -- (2) -- (10) -- (60);
    \draw (5) -- (10);
\end{tikzpicture}
\end{center}

\subsection*{b) }
\begin{gather*}
M_R = 
\begin{bmatrix}
1 & 0 & 1 & 1 & 1 \\
0 & 1 & 1 & 0 & 1 \\
0 & 0 & 1 & 0 & 1 \\
0 & 0 & 0 & 1 & 0 \\
0 & 0 & 0 & 0 & 1 \\
\end{bmatrix}
\end{gather*}

\subsection*{c) }
\begin{gather*}
M_{R_s} = 
\begin{bmatrix}
1 & 0 & 1 & 1 & 1 \\
0 & 1 & 1 & 0 & 1 \\
1 & 1 & 1 & 0 & 1 \\
1 & 0 & 0 & 1 & 0 \\
1 & 1 & 1 & 0 & 1 \\
\end{bmatrix}
\end{gather*}
\begin{gather*}
\forall \+ x \in \{(10, 2), (18, 2), (60, 2), (10, 5), (60, 5), (60, 10)\} (x \in R_s \land x \notin R)
\end{gather*}

\subsection*{d) }
It is not possible to create a total ordering relation with replacing one element, since $2$ and $5$ are unrelated because they are both (relatively) prime.
If we replace $2$, $18$ will be left out of the chain because $5 \ndiv 18$.
If we replace $5$, $10$ and $18$ will be on separate chains because $10 \ndiv 18$.
If we replace any other element, $2$ and $5$ will be on separate chains because $2 \ndiv 5$.
Therefore, it is not possible to create a total ordering relation with replacing just one element.
However, it is possible by removing two elements and adding one.
For example, removing $5$ and $18$, and adding $20$, $30$ or any multiple of $60$ greater than $60$ will work.
Another example is removing $2$ and $18$, and adding $20$, $30$ or any multiple of $60$ greater than $60$.

\end{document}