\documentclass[12pt]{article}
\usepackage[utf8]{inputenc}
\usepackage{float}
\usepackage{amsmath}

\usepackage[hmargin=3cm,vmargin=6.0cm]{geometry}
%\topmargin=0cm
\topmargin=-2cm
\addtolength{\textheight}{6.5cm}
\addtolength{\textwidth}{2.0cm}
%\setlength{\leftmargin}{-5cm}
\setlength{\oddsidemargin}{0.0cm}
\setlength{\evensidemargin}{0.0cm}

\usepackage{braket}

\begin{document}

\addtolength\jot{5pt}

\section*{Student Information } 
%Write your full name and id number between the colon and newline
%Put one empty space character after colon and before newline
Full Name : Murat Bolu \\
Id Number : 2521300 \\

% Write your answers below the section tags
\section*{Answer 1}
Let $\Braket{a_0, a_1, a_2, a_3, \cdots, a_n, \cdots}$ be the generating function $A(x)$. Then, $A(x) = \displaystyle\sum_{n=0}^{\infty} a_n x^n$.
\begin{align*}
\sum_{n=2}^{\infty} a_n x^n &= \sum_{n=2}^{\infty} (3a_{n-1} + 4a_{n-2}) \cdot x^n \\
&= \sum_{n=2}^{\infty} 3a_{n-1} x^n + 4a_{n-2} x^n \\
&= \sum_{n=2}^{\infty} 3a_{n-1} x^n + \sum_{n=2}^{\infty} 4a_{n-2} x^n \\
&= 3 \sum_{n=2}^{\infty} a_{n-1} x^n + 4 \sum_{n=2}^{\infty} a_{n-2} x^n \\
&= 3x \sum_{n=2}^{\infty} a_{n-1} x^{n-1} + 4x^2 \sum_{n=2}^{\infty} a_{n-2} x^{n-2} \\
&= 3x \sum_{n=1}^{\infty} a_n x^n + 4x^2 \sum_{n=0}^{\infty} a_n x^n \\
\sum_{n=2}^{\infty} a_n x^n &= 3x \sum_{n=1}^{\infty} a_n x^n + 4x^2 \sum_{n=0}^{\infty} a_n x^n \\
A(x) - a_1 \cdot x - a_0 &= 3x \cdot (A(x) - a_0) + 4x^2 \cdot A(x) \\
A(x) - a_1 \cdot x - a_0 &= 3x \cdot A(x) - 3x \cdot a_0 + 4x^2 \cdot A(x) \\
A(x) - 3x \cdot A(x) - 4x^2 \cdot A(x) &= a_1 \cdot x + a_0 - 3x \cdot a_0 \\
A(x) \cdot (1 - 3x - 4x^2) &= a_1 \cdot x + a_0 \cdot (1 - 3x) \\
A(x) &= \frac{a_1 \cdot x + a_0 \cdot (1 - 3x)}{1 - 3x - 4x^2} \\
&= \frac{a_1 \cdot x + a_0 \cdot (1 - 3x)}{(1-4x) \cdot (1+x)}
\end{align*}
Plug in values for $a_0$ and $a_1$:
\begin{align*}
A(x) &= \frac{x + 1 \cdot (1 - 3x)}{(1-4x) \cdot (1+x)} \\
&= \frac{1 - 2x}{(1-4x) \cdot (1+x)}
\end{align*}

\noindent
Partial fractions:
\begin{align*}
\frac{1 - 2x}{(1-4x) \cdot (1+x)} &= \frac{\xi}{1-4x} + \frac{\varphi}{1+x} \\
&= \frac{\xi \cdot (1+x) + \varphi \cdot (1-4x)}{(1-4x) \cdot (1+x)} \\
&= \frac{\xi + \xi \cdot x + \varphi - 4 \varphi \cdot x}{(1-4x) \cdot (1+x)} \\
\frac{1 - 2x}{(1-4x) \cdot (1+x)} &= \frac{\xi + \varphi + x \cdot (\xi  - 4 \varphi)}{(1-4x) \cdot (1+x)}
\end{align*}
We can solve the system of linear equations:
\begin{align*}
\xi + \varphi &= 1 \\
\xi -4 \varphi &= -2 \\
5 \varphi &= 3 \\
\varphi &= \tfrac{3}{5} \\
\xi &= \tfrac{2}{5}
\end{align*}
So the equation becomes:
\begin{align*}
A(x) &= \frac{\tfrac{2}{5}}{1-4x} + \frac{\tfrac{3}{5}}{1+x} \\
&= \frac{2}{5} \cdot \frac{1}{1-4x} + \frac{3}{5} \cdot \frac{1}{1+x}
\end{align*}
\begin{align*}
\frac{1}{1-4x} &\leftrightarrow \Braket{4^0, 4^1, 4^2, \cdots, 4^n, \cdots} \\
\frac{2}{5} \cdot \frac{1}{1-4x} &\leftrightarrow \Braket{\tfrac{2}{5} \cdot 4^0, \tfrac{2}{5} \cdot 4^1, \tfrac{2}{5} \cdot 4^2, \cdots, \tfrac{2}{5} \cdot 4^n, \cdots} \\
&\leftrightarrow \Braket{\frac{2}{5}, \frac{8}{5}, \frac{32}{5}, \cdots, \frac{2^{2n+1}}{5}, \cdots}
\end{align*}
\begin{align*}
\frac{1}{1+x} &\leftrightarrow \Braket{1, -1, 1, \cdots, (-1)^n, \cdots} \\
\frac{3}{5} \cdot \frac{1}{1+x} &\leftrightarrow \Braket{\frac{3}{5}, -\frac{3}{5}, \frac{3}{5}, \cdots, \frac{3}{5} \cdot (-1)^n, \cdots}
\end{align*}
\begin{align*}
A(x) &\leftrightarrow \Braket{\frac{2}{5}, \frac{8}{5}, \frac{32}{5}, \cdots, \frac{2^{2n+1}}{5}, \cdots} + \Braket{\frac{3}{5}, -\frac{3}{5}, \frac{3}{5}, \cdots, \frac{3}{5} \cdot (-1)^n, \cdots} \\
A(x) &\leftrightarrow \Braket{1, 1, 7, \cdots, \frac{2^{2n+1} + 3(-1)^n}{5}, \cdots}
\end{align*}
Therefore, $a_n = \dfrac{2^{2n+1} + 3(-1)^n}{5}$.

\section*{Answer 2}
\subsection*{a) }
Let $\Braket{2, 5, 11, 29, 83, 245, \cdots}$ be the generating function $F(x)$.
\begin{align*}
F(x) &\leftrightarrow \Braket{2, 5, 11, 29, 83, 245, \cdots} \\
F(x) - 2 &\leftrightarrow \Braket{0, 3, 9, 27, 81, 243, \cdots} \\
\end{align*}

\subsection*{b) }

\section*{Answer 3}
\subsection*{a) }

\subsection*{b) }

\section*{Answer 4}


\end{document}